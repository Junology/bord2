\documentclass[pdftex,a4paper,12pt]{scrartcl}

\usepackage{amsmath,amssymb,amsthm}
\usepackage{mathtools}
\usepackage{hyperref}
\usepackage[capitalise]{cleveref}
\usepackage{tikz}
\usetikzlibrary{cd}
\usepackage{algpseudocode}

%%%
%%% THEOREM ENVIRONMENTS (amsthm.sty)
%%%
\theoremstyle{plain}
\newtheorem{theorem}{Theorem}[section]
\newtheorem{proposition}[theorem]{Proposition}
\newtheorem{corollary}[theorem]{Corollary}
\newtheorem{lemma}[theorem]{Lemma}

\theoremstyle{definition}
\newtheorem*{definition}{Definition}

\theoremstyle{remark}
\newtheorem{example}[theorem]{Example}
\newtheorem{remark}[theorem]{Remark}
\newtheorem*{notation}{Notation}

%%%
%%% Article data
%%%
\title{Technical Materials}
\author{Jun Yoshida}
\date{\today}

\begin{document}
\maketitle

In this note, we explain some technical materials used in the codes.

\tableofcontents

\section{On the second fundamental forms}

\subsection{Definition}

\begin{definition}
Let $M$ be an orientable $n$-dimensional manifold embedded in the Euclidean space $\mathbb R^{n+1}$.
Then, a \emph{normal vector field} is a section $\mathbf n$ of the vector bundle $\left.T\mathbb R^{n+1}\right|_M\to M$ such that, for each $p\in M$, we have
\begin{equation}
\label{eq:normal-vf}
T\mathbb R^{n+1}= T_pM\oplus \mathbb R\mathbf n(p)
\quad.
\end{equation}
We in addition say that $\mathbf n$ is positive provided the identity \eqref{eq:normal-vf} respects the orientation.
\end{definition}

Note that every $M$ as in the definition above admits a unique normal vector field of constant length $1$; indeed, the standard metrin con $\mathbb R^{n+1}$ and the orientation on $M$ determine a splitting of the following short exact sequence of vector bundles:
\[
0\to TM\to \left.T\mathbb R^{n+1}\right|_M \to \nu M\to 0
\quad,
\]
where $\nu M$ is the normal bundle of the embedding $M\hookrightarrow\mathbb R^{n+1}$.
On the other hand, since $M$ is orientable, the bundle $\nu M$ is orientable and hence a trivial bundle.
Thus, one can choose a nowhere-vanishing section $s:M\to \nu M$, and a unit normal vector field $\mathbf n$ is obtained by normalizing $s$; here is only one choice of $\mathbf n$ so that it is positive.

In the case above, let us denote by $S\mathbb R^{n+1}$ the unit sphere bundle associated to the tangent bundle $\mathbb R^{n+1}$; hence, if $\mathbf n$ is a positive normal vector field on $M$, then the assignment
\[
M \to S\mathbb R^{n+1}
\ ;\quad p\mapsto \frac{\mathbf n(p)}{\|\mathbf n(p)\|}
\]
defines a section of the bundle $\left. S\mathbb R^{n+1}\right|_M\to M$.
Since $S\mathbb R^{n+1}$ has a canonical trivialization, this gives rise to a smooth map $\nu:M\to S^n$, which we call the \emph{Gauss map}.

\begin{definition}
Let $\nu$ be the Gauss map on $M$.
Then, the \emph{second fundamental form} is the bilinear form
\[
\mathbf{II}_M:TM\otimes_M TM\to\underline{\mathbb R}_M
\]
given by
\[
\mathbf{II}_M(v,w)
\coloneqq -\langle d\nu(v),w\rangle
\quad.
\]
\end{definition}

\begin{example}
In case $n=1$, let $\gamma=(\gamma_1,\gamma_2):(-\varepsilon,\varepsilon)\to M\subset\mathbb R^2$ be a regular curve which gives a coordinate around a point $\gamma(0)=p\in M$.
Then, we have a positive normal vector field on the coordinate given by
\[
\gamma^\perp(t)\coloneqq \left(-\frac{d\gamma_2}{dt}(t),\frac{d\gamma_1}{dt}(t)\right)
\quad.
\]
In particular, if $\gamma$ is parametrized by arc-length, $\gamma^\perp$ gives the Gauss map as $\nu\gamma=\gamma^\perp$.
It follows that
\[
d\nu\left(\frac{d\gamma}{dt}(0)\right)
= \frac{d\gamma^\perp}{dt}(0)
= \left(-\frac{d^2\gamma_2}{dt^2}(0),\frac{d^2\gamma_1}{dt^2}(0)\right)
\]
and the second fundamental form is determined by the equation
\[
\mathbf{II}_\gamma\left(\frac{d\gamma}{dt}(0),\frac{d\gamma}{dt}(0)\right)
= \frac{d^2\gamma_2}{dt^2}(0)\frac{d\gamma_1}{dt}(0) - \frac{d^2\gamma_1}{dt^2}(0)\frac{d\gamma_2}{dt}(0)
= \left\langle \frac{d^2\gamma}{dt^2}(0), \gamma^\perp(0)\right\rangle
\quad.
\]
Note that, since $\gamma$ has a constant velocity, namely $\|d\gamma/dt\|\equiv1$, the vector $d^2\gamma/dt^2(0)$ is parallel to $\gamma^\perp$.
Hence, we have
\[
\mathbf{II}_\gamma\left(\frac{d\gamma}{dt}(0),\frac{d\gamma}{dt}(0)\right)\cdot\gamma^\perp(0)
= \frac{d^2\gamma}{dt^2}(0)
\quad,
\]
which is exactly the curvature vector of the curve $\gamma$.
\end{example}

\subsection{Formula for the regular level sets}

We now compute the second fundamental form for the regular level set of a function.
Let $f:\mathbb R^{n+1}\to\mathbb R$ be a smooth function with $0\in\mathbb R$ as a regular value.
We compute the second fundamental form on the submanifold
\[
M\coloneqq\{x\in\mathbb R^{n+1}\mid f(x)=0\}
\quad.
\]
It directly follows from the definition that the following is an exact sequence of vector bundles over $M$:
\[
0\to TM\hookrightarrow \left.T\mathbb R^{n+1}\right|_M \xrightarrow{df} \underline{\mathbb R}_M\to 0
\quad,
\]
where $\underline{\mathbb R}_M$ is the trivial bundle with fiber $\mathbb R$.
In other words, a vector $a_0\frac\partial{\partial x_0}+\dots+a_n\frac\partial{\partial x_n}$ on $\mathbb R^{n+1}$ is tangent to $M$ if and only if the vector
\begin{equation}
\label{eq:df-normal}
df\left(\sum_{i=0}^n a_i\frac\partial{\partial x_i}\right)
= \sum_{i=0}^n a_i\frac{\partial f}{\partial x_i}
\end{equation}
vanishes.
This implies that, for each $\vec x=(x_0,\dots,x_n)\in M$, the vector
\[
df^\ast(\vec x)
\coloneqq \left(\frac{\partial f}{\partial x_0}(\vec x),\dots,\frac{\partial f}{\partial x_n}(\vec x)\right)
\]
is normal to $M$.
We endow $M$ with the orientation so that $\mathbf n_f$ is positive.
Then, for the Gauss map $\nu$, we have
\begin{equation}
\label{eq:f-Gauss}
d_{\vec x}\nu\left(\sum_{i=0}^n a_i\frac\partial{\partial x_i}\right)
= \sum_{i=0}^n a_i \left.\frac{\partial}{\partial x_i}\right|_{\vec x} \frac{df^\ast}{\|df^\ast\|}
\quad.
\end{equation}
Note that the equation \eqref{eq:df-normal} implies that
\begin{equation}
\label{eq:diff-normrec}
\begin{split}
\sum_{i=0}^n a_i\left.\frac{\partial}{\partial x_i}\right|_{\vec x}
\frac1{\|df^\ast\|}
&= \sum_{i=0}^n a_i\left.\frac{\partial}{\partial x_i}\right|_{\vec x}\left(\sum_{j=0}^n\left(\frac{\partial f}{\partial x_j}\right)^2\right)^{-1/2} \\
& = -\frac1{2\|df^\ast(\vec x)\|^3} \sum_{i=0}^n\sum_{j=0}^n 2a_i\frac{\partial f}{\partial x_j}(\vec x)\frac{\partial^2f}{\partial x_i\partial x_j}(\vec x) \\
& = -\frac1{\|df^\ast(\vec x)\|^3}\sum_{j=0}^n \frac{\partial f}{\partial x_j}(\vec x)\left.\frac{\partial}{\partial x_j}\right|_{\vec x} \sum_{i=0}^n a_i\frac{\partial f}{\partial x_i} \\
& = 0
\quad.
\end{split}
\end{equation}
Combining \eqref{eq:f-Gauss} and \eqref{eq:diff-normrec}, we obtain
\begin{equation}
\label{eq:f-II-form}
\mathbf{II}_{\{f=0\}}\left(\sum_{i=0}^na_i\frac{\partial}{\partial x_i},\sum_{j=0}^nb_j\frac\partial{\partial x_j}\right)
= -\frac{1}{\|df^\ast\|}\sum_{i=0}^n\sum_{j=0}^n a_ib_j\frac{\partial^2f }{\partial x_i\partial x_j}
\quad.
\end{equation}
If we denote by $H_f$ the Hessian matrix, then the bilinear form $\mathbf{II}_{\{f=0\}}$ is represented by the matrix $H_f/\|df^\ast\|$.

\begin{example}
\label{ex:II-on-circle}
For a positive real number $r$, define $f:\mathbb R^2\to\mathbb R$ by
\begin{equation}
\label{eq:circle-deffun}
f(x,y)\coloneqq x^2+y^2-r^2
\quad.
\end{equation}
Put $C\coloneqq\{f=0\}$ the zero-set; i.e. the circle of radius $r$.
For each $(x,y)\in C$, we have
\[
df^\ast(x,y) = (2x,2y)
\ ,\quad
\|df^\ast(x,y)\| = \sqrt{4x^2 + 4y^2} = 2r
\quad.
\]
On the other hand, $C$ is parametrized by the map
\[
\gamma:\mathbb R\to C
\ ;\quad \theta\mapsto \left(r\cos\frac\theta{r},r\sin\frac\theta{r}\right)
\]
so that $\|d\gamma/d\theta\|\equiv 1$.
We have
\[
\mathbf{II}_C\left(\frac{d\gamma}{dt}(\theta),\frac{d\gamma}{dt}(\theta)\right)
= \frac1{2r}\left(2\left(-\sin\frac\theta{r}\right)^2 + 2\left(\cos\frac\theta{r}\right)^2\right) = \frac1{r}
\quad.
\]
Note that the vector
\begin{equation}
\label{eq:II-curvature}
\mathbf{II}_C\left(\frac{d\gamma}{dt}(\theta),\frac{d\gamma}{dt}(\theta)\right)\cdot \frac{df^\ast}{\|df^\ast\|}
= -\frac{1}{r^2}\gamma(\theta)
\end{equation}
equals the curvature vector of $C$.
\end{example}

\begin{remark}
If a codimension $1$ submanifold $M\subset\mathbb R^{n+1}$ is defined as a zero-set of a function $f$, then the orientation on $M$ does depend on $f$.
For example, to define the same submanifold $C$ as in \cref{ex:II-on-circle}, one can take a function $g:\mathbb R^2\to\mathbb R^2$ given by
\[
g(x,y)\coloneqq -x^2-y^2+r^2
\]
instead of $f$ in \eqref{eq:circle-deffun}.
Then, the induced orientation on $C$ is reversed; so we have
\[
\mathbf{II}_{\{g=0\}}\left(\frac{d\gamma}{dt}(\theta),\frac{d\gamma}{dt}(\theta)\right)
= -\frac1{r}
\quad.
\]
In contrast, since $df^\ast = -dg^\ast$, the curvature vector \eqref{eq:II-curvature} is invariant of the choice of the defining function.
\end{remark}

More generally, we have the following result in the $2$-dimensional cases.

\begin{proposition}
\label{prop:f-curvature}
Let $f:\mathbb R^2\to\mathbb R$ be a smooth function with $0\in\mathbb R$ being a regular value; put $C\coloneqq\{(x,y)\mid f(x,y)=0\}$.
Then, the curvature vector $\kappa$ of $C$ is, as a function on $C$, given by
\[
\kappa
= -\left(
\left(\frac{\partial f}{\partial y}\right)^2\frac{\partial^2 f}{\partial x^2}
-2\frac{\partial f}{\partial x}\frac{\partial f}{\partial y}\frac{\partial^2 f}{\partial x\partial y}
+\left(\frac{\partial f}{\partial x}\right)^2\frac{\partial^2 f}{\partial y^2}
\right)\cdot
\frac{df^\ast}{\|df^\ast\|^4}
\]
\end{proposition}
\begin{proof}
We set
\[
df^\perp(x,y)
\coloneqq \left(-\frac{\partial f}{\partial y}(x,y), \frac{\partial f}{\partial x}(x,y)\right)
\quad.
\]
It is easily verify that, if $(x,y)\in C$, then $df^\perp(x,y)$ is a tangent vector at $(x,y)$.
We also have $\|df^\perp(x,y)\|=\|df^\ast(x,y)\|$.
Hence, using the formula \eqref{eq:f-II-form}, we obtain
\[
\begin{split}
\kappa
&= -\mathbf{II}_C\left(\frac{df^\perp}{\|df^\perp\|},\frac{df^\perp}{\|df^\perp\|}\right)\cdot\frac{df^\ast}{\|df^\ast\|} \\
&= -\left(
\left(\frac{\partial f}{\partial y}\right)^2\frac{\partial^2 f}{\partial x^2}
-2\frac{\partial f}{\partial x}\frac{\partial f}{\partial y}\frac{\partial^2 f}{\partial x\partial y}
+\left(\frac{\partial f}{\partial x}\right)^2\frac{\partial^2 f}{\partial y^2}
\right)\cdot
\frac{df^\ast}{\|df^\ast\|^4}
\quad.
\end{split}
\]
\end{proof}

\begin{example}
Let $f:\mathbb R^2\to\mathbb R$ be a quadratic function of the form
\[
f(x,y)=ax^2+by^2-c
\]
with $ab\neq 0$ and $c\neq 0$.
Then,
\[
df^\ast(x,y) = (2ax, 2by)
\ ,\quad
\|df^\ast(x,y)\| = 2\sqrt{a^2x^2+b^2y^2}
\quad;
\]
so in particular, $0\in\mathbb R$ is a regular value of $f$.
By virtue of \cref{prop:f-curvature}, we obtain
\[
\kappa(x,y)
= -\frac{8ab^2y^2+8a^2bx^2}{16(a^2x^2+b^2y^2)^2}(2ax,2by)
= -\frac{abc}{(a^2x^2+b^2y^2)^2}(ax,by)
\quad.
\]
For example, if $a=b=1$ and $c=r^2$, this agrees with the computation in \cref{ex:II-on-circle}.
\end{example}


\section{Quadratic B\'ezier triangles}
\label{sec:qbezier}

\subsection{Definition}

Let $V$ be a (finite dimensional) real vector space, so there is a canonical identification $TV=V\times V$.
We write
\[
\Delta^2\coloneqq\{(t_0,t_1,t_2)\in\mathbb R^3\mid t_0+t_1+t_2=1,\,t_i\ge0\ \text{for $i=0,1,2$}\}
\]
the (geometric) $2$-simplex.

\begin{definition}
A \emph{quadratic B\'ezier triangle} in $V$ is a map $p:\Delta^2\to V$ of the form
\[
p(t_0,t_1,t_2)
= \sum_{i=0}^2 t_i^2 v^{(i)}+2\sum_{\{i,j,k\}=\{0,1,2\}} t_i t_je^{(k)}
\]
for six points $v^{(0)},v^{(1)},v^{(2)},e^{(0)},e^{(1)},e^{(2)}\in V$, which are called the \emph{control points} of $p$.
\end{definition}

It is easily seen that
\[
p(1,0,0) = v^{(0)}
\ ,\quad p(0,1,0) = v^{(1)}
\ ,\quad p(0,0,1) = v^{(2)}
\ ,
\]
while the image $p(\Delta^2)$ may not contain the points $e^{(0)}$, $e^{(1)}$, and $e^{(2)}$.

\begin{lemma}
A quadratic B\'ezier triangle restricts to a quadratic B\'ezier curve on each edge of $\Delta^2$.
\end{lemma}
\begin{proof}
Note that an edge of $\Delta^2$, namely
\[
\partial_i\Delta^2\coloneqq \{(t_0,t_1,t_2)\in\Delta^2\mid t_i=0\}
\]
for $i=0,1,2$, is parametrized so that, with cyclic indices,
\[
t_i=0
,\,
t_{i+1}=t
,\,
t_{i+2}=1-t
\quad.
\]
In this point of view, we have
\[
\left.p\right|_{\partial_i\Delta^2}(t)
= t^2v^{(i+1)} + 2t(1-t) e^{(i)} + (1-t)^2 v^{(i+2)}
\quad.
\]
\end{proof}

\subsection{Tangent space}

We compute the tangent spaces of quadratic B\'ezier triangles.
We fix a quadratic B\'ezier triangle $p:\Delta^2\to V$ with control points $v^{(0)},v^{(1)},v^{(2)},e^{(0)},e^{(1)},e^{(2)}\in V$.

First observe that, a vector
\[
X=a\frac\partial{\partial t_0}+b\frac\partial{\partial t_1}+c\frac\partial{\partial t_2}
\]
on a point of $\Delta^2$ is tangent to $\Delta^2$ if and only if $a+b+c=0$.
We in particular set
\[
X^{ij}\coloneqq  \frac12\left(\frac\partial{\partial t_i}-\frac\partial{\partial t_j}\right)
\quad.
\]
Then, for each point $\vec t\in\Delta^2$, the tangent space $T_{\vec t}\Delta^2$ is spanned by two vectors $X^{ij}$ and $X^{ik}$ provided $\{i,j,k\}=\{0,1,2\}$.
On the other hand, we have
\[
p_\ast\left(\frac\partial{\partial t_i}\right)
= 2t_iv^{(i)}+2t_je^{(k)}+2t_ke^{(j)}
\]
and hence
\begin{equation}
\label{eq:dpX}
\begin{split}
p_\ast(X^{ij})
&= t_iv^{(i)}-t_jv^{(j)}+(t_j-t_i)e^{(k)}+t_k(e^{(j)}-e^{(i)}) \\
&= t_i(v^{(i)}-e^{(k)})+t_j(e^{(k)}-v^{(j)})+t_k(e^{(j)}-e^{(i)})
\quad.
\end{split}
\end{equation}
Putting $w_{ij}\coloneqq e^{(k)}-v^{(i)}$, one also obtains
\begin{equation}
\label{eq:dpX-inw}
p_\ast(X^{ij})
= -t_iw_{ij}+t_jw_{ji}+t_k(w_{ki}-w_{kj})
\quad.
\end{equation}

\subsection{Singular loci}

We are interested in the singular locus of a quadratic B\'ezier triangle $p:\Delta^2\to V$ for $\dim V\ge 2$.
In view of the previous section, a point $\vec t=(t_0,t_1,t_2)\in\Delta^2$ is a critical point of $p$ if and only if two vectors $p_\ast(X^{ij})$ and $p_\ast(X^{ik})$ are in parallel for $\{i,j,k\}=\{0,1,2\}$.
Using the exterior product $V\wedge V$, this is equivalent to the equation
\[
p_\ast(X^{ij})\wedge p_\ast(X^{ik})=0
\quad.
\]
On the other hand, by virtue of the equation \eqref{eq:dpX-inw}, we have
\begin{equation}
\label{eq:singlocus-def}
\begin{split}
& p_\ast(X^{ij})\wedge p_\ast(X^{ik}) \\
&= \left(-t_iw_{ij}+t_jw_{ji}+t_k(w_{ki}-w_{kj})\right)\wedge\left(-t_iw_{ik}+t_j(w_{ji}-w_{jk})+t_kw_{ki}\right) \\
&=
\begin{multlined}[t]
t_i^2 w_{ij}\wedge w_{ik}-t_j^2w_{ji}\wedge w_{jk}-t_k^2w_{kj}\wedge w_{ki}
- t_it_j\left(w_{ij}\wedge(w_{ji}-w_{jk})+w_{ji}\wedge w_{ik}\right) \\
- t_jt_k\left(w_{ki}\wedge w_{jk}+w_{kj}\wedge(w_{ji}-w_{jk})\right)
- t_it_k\left(w_{ij}\wedge w_{ki}+(w_{ki}-w_{kj})\wedge w_{ik}\right)
\quad.
\end{multlined}
\end{split}
\end{equation}
It follows that the quadratic form \eqref{eq:singlocus-def} together with the linear equation $t_i+t_j+t_K=1$ defines the singular locus of $p$.
Note that, if $\dim V=n$, then $V\wedge V$ is of dimension $n(n-1)/2$; so we have $n(n-1)/2$ defining polynomials.
In particular, there is generically no critical point of $p$ except in the case $n=2$ where they form a $1$-dimensional submanifold of $\Delta^2$.


\section{On quadratic curves}

\subsection{Definition}

Recall that a \emph{quadratic form} on $\mathbb R^n$ is nothing but a real homogeneous polynomial of degree $2$ with $n$ variables.
More precisely, it is a polynomial of the form
\begin{equation}
\label{eq:quad-form-alph}
q(x_1,\dots,x_n)
= \sum_{i=1}^n A_ix_i^2+2\sum_{1\le i<j\le n} B_{ij}x_ix_j
\quad.
\end{equation}
For a quadratic form $q$ on $\mathbb R^n$, we define the associated matrix $M_q$ to be
\[
\begin{bmatrix}
A_1 & B_{12} & \cdots & B_{1n} \\
B_{12} & A_2 & \cdots & B_{2n} \\
\vdots & \vdots & \ddots & \vdots \\
B_{1n} & B_{2n} & \cdots & A_n
\end{bmatrix}
\quad.
\]
Then, we can write
\begin{equation}
\label{eq:quad-form-mat}
q(x_1,\dots,x_n)
= \vec x^{\mathsf T} M_q\vec x
=
\begin{bmatrix}
x_1 & \cdots & x_n
\end{bmatrix}
M_q
\begin{bmatrix}
x_1 \\ \vdots \\ x_n
\end{bmatrix}
\quad.
\end{equation}

\begin{definition}
A quadratic form $q$ on $\mathbb R^n$ is said to be \emph{non-degenerate} (resp. \emph{degenerate}) if $\det M_q\neq 0$ (resp. $\det M_q=0$).
\end{definition}

\begin{lemma}
If $q$ is a non-degenerate quadratic form on $\mathbb R^n$, then for the map
\[
f_q:\mathbb R^{n-1}\to\mathbb R\ ;\quad (y_1,\dots,y_{n-1}) \mapsto q(y,\dots ,y_{n-1},1)
\quad,
\]
$0\in\mathbb R$ is a regular value of $f_q$.
\end{lemma}
\begin{proof}
We prove the contraposition; in particular, we show that if $\vec y=(y_1,\dots,y_{n-1})\in\mathbb R^{n-1}$ is a critical point of $f_q$ with $f_q(\vec y)=0$, then the vector $(y_1,\dots,y_{n-1},1)$ belongs to the kernel of $M_q$.

The direct computation shows that the total derivative $df:\mathbb R^{n-1}\to T^\ast\mathbb R^{n-1}$ is given by
\[
df_q(y_1,\dots,y_{n-1})
= \sum_{i=1}^{n-1}\left(\vec e^{(i)\mathsf T} M_q\widehat y+\widehat y^{\mathsf T} M_q e_i\right) dy_i \\
= 2 \sum_{i-1}^{n-1} \vec e^{(i)\mathsf T}M_q\widehat y\, dy_i
\quad,
\]
here $\widehat y=(y_1,\dots,y_{n-1},1)$ and $\vec e^{(i)}=(0,\dots,\overset{\substack{i\\\smash\smile}}{1},\dots,0)$ seen as column vectors.
Under the canonical identification $T^\ast\mathbb R^{n-1}\cong \mathbb R^{n-1}\times\mathbb R^{n-1}$, it turns out that $df$ can be seen as the map given by
\begin{equation}
\label{eq:prf:fq-diff}
df_q(y_1,\dots,y_{n-1})
= 2M_q'\vec y+ 2v_q
\quad,
\end{equation}
where $\vec y=(y_1,\dots,y_{n-1},0)$ and $M_q'$ and $v_q$ are defined so that
\[
M_q =
\left[
\begin{array}{c|c}
  M_q' & v_q  \rule[-1.5ex]{0pt}{4ex}\\\hline\rule[-1.5ex]{0pt}{4ex}
  v_q^{\mathsf T} & A_n
\end{array}
\right]
\quad.
\]
On the other hand, we have
\[
\begin{split}
f_q(y_1,\dots,y_{n-1})
&= q(y_1,\dots,y_{n-1},1) \\
&= \widehat y^{\mathsf T} M_q\widehat y \\
&= \vec y^{\mathsf T} M_q'\vec y + 2v_q^{\mathsf T}\vec y + A_n
\quad.
\end{split}
\]
Hence, for $(y_1,\dots,y_{n-1})\in\mathbb R^{n-1}$ with $df(y_1,\dots,y_{n-1})=0$, we have
\begin{equation}
\label{eq:prf:fq-critv}
f_q(y_1,\dots,y_{n-1})
= v_q^{\mathsf T}\vec y + A_n
\quad.
\end{equation}
Combining \eqref{eq:prf:fq-diff} and \eqref{eq:prf:fq-critv}, we obtain $M_q\widehat y=0$ provided $df_q(y_1,\dots,y_{n-1})=f_q(y_1,\dots,y_{n-1})=0$.
This is exactly what we want to show.
\end{proof}

\begin{definition}
A \emph{(real) quadratic curve} is a curve $C$ in $\mathbb R^2$ such that
\[
C=\{(x,y)\in\mathbb R^2\mid q(x,y,1)=0\}
\]
for a quadratic form $q(x,y,z)$ on $\mathbb R^3$.
\end{definition}

\begin{remark}
It is easily seen that a quadratic form actually defines an algebraic function on the projective space $\mathbb P^n$.
One may sometimes use the word ``quadratic curves'' to refer algebraic subsets of the projective space $\mathbb P^2$ defined by quadratic forms of three variables.
In such cases, ``quadratic curves'' are always compact while ours might not.
\end{remark}

\subsection{The srandard forms}

\begin{proposition}
\label{prop:quad-standard}
Let $q$ be a quadratic form on $\mathbb R^n$ with matrices $M_q$ and $M'_q$ as in the previous section.
Suppose both the determinants $\det M_q$ and $\det M_q$ are non-zero.
Then, there is an orthogonal matrix $P\in\mathcal O(n-1)$ such that
\begin{gather}
\label{eq:M'-diag}
M'_q = P
\begin{bmatrix}
\lambda_1 && O \\
& \ddots & \\
O && \lambda_{n-1}
\end{bmatrix}
P^{\mathsf T}
\qquad\text{with $\lambda_i\in\mathbb R$}
\quad;
\\
\label{eq:q-standard}
q(y_1,\dots,y_{n-1},1)
= \sum_{i=1}^{n-1} \lambda_i\left(Y_i+\frac{\widetilde B_{i n}}{\lambda_i}\right)^2 + \frac{\det M_q}{\det M'_q}
\quad,
\end{gather}
where
\[
\begin{bmatrix}
Y_1\\\vdots\\ Y_{n-1}
\end{bmatrix}
\coloneqq P^{\mathsf T}
\begin{bmatrix}
y_1\\\vdots\\ y_{n-1}
\end{bmatrix}
\quad, \qquad
\begin{bmatrix}
\widetilde B_{1n} \\ \vdots \\ \widetilde B_{(n-1)n}
\end{bmatrix}
\coloneqq P^{\mathsf T}
\begin{bmatrix}
B_{1n} \\ \vdots \\ B_{(n-1)n}
\end{bmatrix}
\]
\end{proposition}
\begin{proof}
Since $M'_q$ is a symmetric matrix, we can take an orthogonal matrix $P$ satisfying \eqref{eq:M'-diag}.
Then, we have
\[
\begin{split}
& q(y_1,\dots,y_{n-1},1) \\
&=
\begin{bmatrix}
y_1 & \cdots & y_{n-1} & 1
\end{bmatrix}
\left[
\begin{array}{c|c}
  P
  \begin{bmatrix}
  \lambda_1 && \\
  & \ddots & \\
  && \lambda_{n=1}
  \end{bmatrix}
     P^{\mathsf T}
  &
    \begin{matrix}
    B_{1n} \\ \vdots \\ B_{(n-1)n}
    \end{matrix}
  \\\hline
  \begin{matrix}
  B_{1n} & \cdots & B_{(n-1)n}
  \end{matrix}
      & A_n
\end{array}
\right]
\begin{bmatrix}
y_1 \\ \vdots \\ y_{n-1} \\ 1
\end{bmatrix}
\\
&=
\begin{bmatrix}
Y_1 & \cdots & Y_{n-1} & 1
\end{bmatrix}
\left[
\begin{array}{c|c}
  \begin{matrix}
  \lambda_1 && \\ & \ddots & \\ && \lambda_{n-1}
  \end{matrix}
             & P^{\mathsf T}
               \begin{bmatrix}
               B_{1n} \\ \vdots \\ B_{(n-1)n}
               \end{bmatrix}
  \\\hline
  \begin{bmatrix}
  B_{1n} & \cdots & B_{(n-1)n}
  \end{bmatrix} P
                 & A_n
\end{array}
\right]
\begin{bmatrix}
Y_1 \\ \vdots \\ Y_{n-1} \\ 1
\end{bmatrix}
\\
&= 
\begin{bmatrix}
Y_1 & \cdots & Y_{n-1} & 1
\end{bmatrix}
\begin{bmatrix}
\lambda_1  && &\widetilde B_{1n} \\
& \ddots && \vdots \\
&& \lambda_{n-1} & \widetilde B_{(n-1)n} \\
\widetilde B_{1n} & \cdots & \widetilde B_{(n-1)n} & A_n
\end{bmatrix}
\begin{bmatrix}
Y_1 \\ \vdots \\ Y_{n-1} \\ 1
\end{bmatrix}
\\
&= \sum_{i=1}^{n-1} (\lambda_i Y_i^2 + 2\widetilde B_{in}Y_i) + A_n
\\
\end{split}
\]
We denote by $\widetilde M_q$ the middle matrix in the fourth line.
Then, we have $\det M_q = \det \widetilde M_q$ and $\det M'_q =\lambda_1\dots\lambda_{n-1}$.
The asumption $\det M'_q=0$ guarantees that all of $\lambda_i$'s are non-zero.
Hence, we obtain
\[
\begin{split}
q(y_1,\dots,y_{n-1},1)
&= \sum_{i=1}^{n-1} \lambda_i\left(Y_i + \frac{\widetilde B_{in}}{\lambda_i}\right)^2
+ A_n-\sum_{i=1}^{n-1}\frac{B_{in}^2}{\lambda_i} \\
&= \sum_{i=1}^{n-1} \lambda_i\left(Y_i + \frac{\widetilde B_{in}}{\lambda_i}\right)^2 + \frac{\det M_q}{\det M'_q}
\end{split}
\]
as required.
\end{proof}

\begin{corollary}
Let $q$ be a quadratic form on $\mathbb R^n$ with $\det M_q\neq 0$ and $\det M'_q\neq 0$.
Set $\Lambda_-\subset\mathbb R$ the set of those eigen values of $M'_q$ which have different signs from the number $(\det M_q)/(\det M'_q)$, and for each $\lambda\in\Lambda_-$, fix an eigenvector $v_\lambda\in\mathbb R^{n-1}$ of $M'_q$ associated to $\lambda$.
We define an affine subspace $H\subset\mathbb R^{n-1}$ by
\[
H
\coloneqq\left\{
\vec y\in\mathbb R^{n-1}\mid \forall\lambda\in\Lambda_-:\langle v_\lambda,M'_q\vec y+\vec B_n\rangle=0
\right\}
\quad,
\]
where $\vec B_n \coloneqq (B_{11},\dots,B_{(n-1)n})$.
Then, we have
\[
H\cap\{(y_1,\dots,y_{n-1})\in\mathbb R^{n-1}\mid q(y_1,\dots,y_{n-1},1)=0\}
= \varnothing
\quad.
\]
\end{corollary}
\begin{proof}
Let $\lambda_1\le\dots\lambda_{n-1}$ be the eigen values of $M'_q$ and $v^{(1)},\dots,v^{(n-1)}\in\mathbb R^{n-1}$ associated unit eigenvectors which are mutually orthogonal.
Then, the matrix $M'_q$ is diagonalized by the orthogonal matrix
\[
P =
\begin{bmatrix}
v^{(1)} & \cdots & v^{(n-1)}
\end{bmatrix}
\quad.
\]
Set $Y_i$ and $\widetilde B_i$ as in \cref{prop:quad-standard}, so we have
\[
Y_i = \langle v^{(i)},\vec y\rangle
\ ,\quad \widetilde B_i = \langle v^{(i)},\vec B_n\rangle
\]
and hence
\[
\langle v^{(i)},M'_q\vec y + \vec B_n\rangle
= \langle M'_q v^{(i)},\vec y\rangle + \langle v^{(i)},\vec B_n\rangle
= \lambda_i Y_i + \widetilde B_n
\quad.
\]
It follows that, if $\vec y\in H$, then all the terms in the sum \eqref{eq:q-standard} have the same sign, which implies the result.
\end{proof}

\begin{example}
Define $q:\mathbb R^3\to\mathbb R$ by
\[
q(x,y,z)
\coloneqq 2(x-z)(y-z) - 2z^2 = 2xy - 2xz - 2yz
\quad.
\]
The associated matrices are given by
\[
M_q =
\begin{bmatrix}
0 & 1 & -1 \\
1 & 0 & -1 \\
-1 & -1 & 0
\end{bmatrix}
\quad,\qquad
M'_q =
\begin{bmatrix}
0 & 1 \\ 1 & 0
\end{bmatrix}
\quad.
\]
It turns out that the eigenvalues of $M'_q$ are $1$ and $-1$ for which associated eigenvectors are given by
\[
v_\pm =
\begin{bmatrix}
1 \\ \pm 1
\end{bmatrix}
\quad.
\]
We also have
\[
\det M_q = 2
\ ,\quad
\det M'_q = -1
\quad.
\]
Now, we set
\[
\begin{split}
H
&\coloneqq\left\{
(x,y)\in\mathbb R^2
\;\middle|\;
\left\langle v_+,M'_q\begin{bmatrix} x \\ y \end{bmatrix} + \begin{bmatrix} -1 \\ -1 \end{bmatrix}\right\rangle
\right\}
\\
&= \left\{
(x,y)\in\mathbb R^2
\;\middle|\;
x+y-2 = 0
\right\}
\end{split}
\quad,
\]
then $H$ does not intersect to the zero-set $\{(x,y)\mid q(x,y,1)=0\}$, which is a hyperbolic curve.
\end{example}

Note that the equation $q(\vec y,1)=0$ is invariant under reflections along the eigenvectors of $M'_q$.
In particular, if $H$ is of codimension $1$, then it divides the zero-set into two connected components.

\subsection{Intersections with segments}

We next discuss the intersections of quadratic curves with line segments.
Recall that, for two points $v,w\in\mathbb R^{n-1}$ in the Euclidean space, the line segment connecting them is given as the function $\gamma_{vw}:[0,1]\to\mathbb R^{n-1}$ with
\[
\gamma_{vw}(t)\coloneqq (1-t)v + tw
\quad.
\]

\begin{lemma}
Let $q$ be a quadratic form on $\mathbb R^n$ and $v,w\in\mathbb R^{n-1}$.
Then, the line segment $\gamma_{vw}$ intersects with the set
\[
Q\coloneqq \left\{(y_1,\dots,y_{n-1})\mid q(y_1,\dots,y_{n-1},1)=0\right\}
\]
at a parameter $t\in[0,1]$ if and only if $t$ is the solution of the following equation:
\[
\left((w-v)^{\mathsf T}M'_q(w-v)\right)t^2
+ 2\left((w-v)^{\mathsf T}(M'_q v + v_q)\right)t
+ \widehat v^{\mathsf T}M_q \widehat v
= 0
\quad,
\]
where $M_q$, $M'_q$, and $v_q$ are defined as in the previous section.
\end{lemma}
\begin{proof}
We put $f_q(y_1,\dots,y_{n-1})\coloneqq q(y_1,\dots,y_{n-1},1)$, then we have to solve the equation $f_q(\gamma_{vw}(t))=0$.
Since $f_q$ is a polynomial of degree at most $2$, computing the Taylor series of the composition $f_q\gamma_{vw}$, we obtain
\[
\begin{split}
&[f_q\circ\gamma_{vw}](t) \\
&= f_q(v+t(w-v)) \\
&= f_q(v)
+ t\cdot\left(\left.\frac{d}{dt}\right|_{t=0} f_q(v+t(w-v))\right)
+ \frac{t^2}{2}\left(\left.\frac{d^2}{dt^2}\right|_{t=0} f_q(v+t(w-v))\right)
\quad.
\end{split}
\]
On the other hand, we have
\[
(\gamma_{vw})_\ast\left(\frac{d}{dt}\right)
= \sum_{i=1}^{n-1} (v_i-w_i)\frac\partial{\partial y_i}
\quad.
\]
Thus, combining with the equation \eqref{eq:prf:fq-diff}, one obtains the result.
\end{proof}

\subsection{Algorithm}

Fix a quadratic form $q(x,y,z)$ on $\mathbb R^3$.
In this section, we discuss an algorithm rendering the quadratic curve $C=\{(x,y)\mid q(x,y,1)=0\}$ with quadratic B\'ezier curvees.
The following is a pseudo-code for our approach:
\begin{algorithmic}[1]
\Procedure{Render-Quad-Curve}{$q$:\:quadratic form, $X$:\:canvas}
\State $T\gets$ sufficiently fine triangualation of $X$
\ForAll{$\Delta\in T$}
\State $S\gets$ \Call{Segments}{$q$,$\Delta$}
\ForAll{$(v,w)\in S$}
\State $c\gets \operatorname{solve}(d_vq(x-v)=d_wq(x-w)=0)$
\State $\operatorname{qbezier}(v,c,w)$
\EndFor
\EndFor
\EndProcedure
\end{algorithmic}
In the function \textsc{Segments}, we compute the pairs $(v,w)$ such that
\begin{itemize}
  \item $v,w\in\partial\Delta\cap\{f_q=0\}$;
  \item $v$ and $w$ belong to the same connected component of $\Delta\cap\{f_q=0\}$.
\end{itemize}
We implement it as follows.
Note that, for two totally ordered sets $P$ and $Q$, we denote by $P\star Q$ the join of them.
\begin{algorithmic}[1]
\Procedure{Segments}{$q$:\:quadratic form, $\Delta$:\:$2$-dim.~polygon}
\State $I\gets\varnothing$ as a totally ordered set
\ForAll{$v\in\operatorname{Vertex}(\Delta)$}
\If{$q(v)=0$}
\State $I\gets I\star\{v\}$
\EndIf
\EndFor
\ForAll{$e\in\operatorname{Edge}(\Delta)$}
\State $I\gets I\star(\operatorname{Interior}(e)\cap \{f_q=0\})$
\EndFor
\State \Return \Call{Pairs}{$q$, $I$}
\EndProcedure

\Procedure{Pairs}{$q$:\:a quadratic form, $I$:\:a set of points in $\{f_q=0\}$}
\If{$\#I\le 1$}
\Comment{Nothing can form a pair}
\State \Return $\varnothing$
\ElsIf{$\#I=2$}
\Comment{A connected pair is found}
\State Let $\{(v,w)\}\gets I$
\State \Return $\{(v,w)\}$
\EndIf
\State $v\gets\operatorname{Head}(I)$
\Comment{Fix a reference point}
\State $I_+\gets \{w\in I\mid d_vq(w-v)>0\}$
\Comment{the subset of points lying on the ``right'' of $v$}
\State $I_-\gets \{w\in I\mid d_vq(w-v)<0\}$
\Comment{the subset of points lying on the ``left'' of $v$}
\State \Return \Call{Pairs}{$q$, $I_+\star\{v\}$} $\cup$ \Call{Pairs}{$q$, $I_-\star\{v\}$}
\Comment{Make pairs on the left and the right respectively}
\EndProcedure
\end{algorithmic}

\end{document}
