\documentclass[pdftex,a4paper,12pt]{scrartcl}

\usepackage{amsmath,amssymb,amsthm}
\usepackage{mathtools}
\usepackage{tikz}
\usetikzlibrary{cd}
\usepackage{algpseudocode}

%%%
%%% THEOREM ENVIRONMENTS (amsthm.sty)
%%%
\theoremstyle{plain}
\newtheorem{theorem}{Theorem}[section]
\newtheorem{proposition}[theorem]{Proposition}
\newtheorem{corollary}[theorem]{Corollary}
\newtheorem{lemma}[theorem]{Lemma}

\theoremstyle{definition}
\newtheorem*{definition}{Definition}

\theoremstyle{remark}
\newtheorem{example}[theorem]{Example}
\newtheorem{remark}[theorem]{Remark}
\newtheorem*{notation}{Notation}

%%%
%%% Article data
%%%
\title{Technical Materials}
\author{Jun Yoshida}
\date{\today}

\begin{document}
\maketitle

In this note, we explain some technical materials used in the codes.

\section{Quadratic B\'ezier triangles}
\label{sec:qbezier}

\subsection{Definition}

Let $V$ be a (finite dimensional) real vector space, so there is a canonical identification $TV=V\times V$.
We write
\[
\Delta^2\coloneqq\{(t_0,t_1,t_2)\in\mathbb R^3\mid t_0+t_1+t_2=1,\,t_i\ge0\ \text{for $i=0,1,2$}\}
\]
the (geometric) $2$-simplex.

\begin{definition}
A \emph{quadratic B\'ezier triangle} in $V$ is a map $p:\Delta^2\to V$ of the form
\[
p(t_0,t_1,t_2)
= \sum_{i=0}^2 t_i^2 v^{(i)}+2\sum_{\{i,j,k\}=\{0,1,2\}} t_i t_je^{(k)}
\]
for six points $v^{(0)},v^{(1)},v^{(2)},e^{(0)},e^{(1)},e^{(2)}\in V$, which are called the \emph{control points} of $p$.
\end{definition}

It is easily seen that
\[
p(1,0,0) = v^{(0)}
\ ,\quad p(0,1,0) = v^{(1)}
\ ,\quad p(0,0,1) = v^{(2)}
\ ,
\]
while the image $p(\Delta^2)$ may not contain the points $e^{(0)}$, $e^{(1)}$, and $e^{(2)}$.

\begin{lemma}
A quadratic B\'ezier triangle restricts to a quadratic B\'ezier curve on each edge of $\Delta^2$.
\end{lemma}
\begin{proof}
Note that an edge of $\Delta^2$, namely
\[
\partial_i\Delta^2\coloneqq \{(t_0,t_1,t_2)\in\Delta^2\mid t_i=0\}
\]
for $i=0,1,2$, is parametrized so that, with cyclic indices,
\[
t_i=0
,\,
t_{i+1}=t
,\,
t_{i+2}=1-t
\quad.
\]
In this point of view, we have
\[
\left.p\right|_{\partial_i\Delta^2}(t)
= t^2v^{(i+1)} + 2t(1-t) e^{(i)} + (1-t)^2 v^{(i+2)}
\quad.
\]
\end{proof}

\subsection{Tangent space}

We compute the tangent spaces of quadratic B\'ezier triangles.
We fix a quadratic B\'ezier triangle $p:\Delta^2\to V$ with control points $v^{(0)},v^{(1)},v^{(2)},e^{(0)},e^{(1)},e^{(2)}\in V$.

First observe that, a vector
\[
X=a\frac\partial{\partial t_0}+b\frac\partial{\partial t_1}+c\frac\partial{\partial t_2}
\]
on a point of $\Delta^2$ is tangent to $\Delta^2$ if and only if $a+b+c=0$.
We in particular set
\[
X^{ij}\coloneqq  \frac12\left(\frac\partial{\partial t_i}-\frac\partial{\partial t_j}\right)
\quad.
\]
Then, for each point $\vec t\in\Delta^2$, the tangent space $T_{\vec t}\Delta^2$ is spanned by two vectors $X^{ij}$ and $X^{ik}$ provided $\{i,j,k\}=\{0,1,2\}$.
On the other hand, we have
\[
p_\ast\left(\frac\partial{\partial t_i}\right)
= 2t_iv^{(i)}+2t_je^{(k)}+2t_ke^{(j)}
\]
and hence
\begin{equation}
\label{eq:dpX}
\begin{split}
p_\ast(X^{ij})
&= t_iv^{(i)}-t_jv^{(j)}+(t_j-t_i)e^{(k)}+t_k(e^{(j)}-e^{(i)}) \\
&= t_i(v^{(i)}-e^{(k)})+t_j(e^{(k)}-v^{(j)})+t_k(e^{(j)}-e^{(i)})
\quad.
\end{split}
\end{equation}
Putting $w_{ij}\coloneqq e^{(k)}-v^{(i)}$, one also obtains
\begin{equation}
\label{eq:dpX-inw}
p_\ast(X^{ij})
= -t_iw_{ij}+t_jw_{ji}+t_k(w_{ki}-w_{kj})
\quad.
\end{equation}

\subsection{Singular loci}

We are interested in the singular locus of a quadratic B\'ezier triangle $p:\Delta^2\to V$ for $\dim V\ge 2$.
In view of the previous section, a point $\vec t=(t_0,t_1,t_2)\in\Delta^2$ is a critical point of $p$ if and only if two vectors $p_\ast(X^{ij})$ and $p_\ast(X^{ik})$ are in parallel for $\{i,j,k\}=\{0,1,2\}$.
Using the exterior product $V\wedge V$, this is equivalent to the equation
\[
p_\ast(X^{ij})\wedge p_\ast(X^{ik})=0
\quad.
\]
On the other hand, by virtue of the equation \eqref{eq:dpX-inw}, we have
\begin{equation}
\label{eq:singlocus-def}
\begin{split}
& p_\ast(X^{ij})\wedge p_\ast(X^{ik}) \\
&= \left(-t_iw_{ij}+t_jw_{ji}+t_k(w_{ki}-w_{kj})\right)\wedge\left(-t_iw_{ik}+t_j(w_{ji}-w_{jk})+t_kw_{ki}\right) \\
&=
\begin{multlined}[t]
t_i^2 w_{ij}\wedge w_{ik}-t_j^2w_{ji}\wedge w_{jk}-t_k^2w_{kj}\wedge w_{ki}
- t_it_j\left(w_{ij}\wedge(w_{ji}-w_{jk})+w_{ji}\wedge w_{ik}\right) \\
- t_jt_k\left(w_{ki}\wedge w_{jk}+w_{kj}\wedge(w_{ji}-w_{jk})\right)
- t_it_k\left(w_{ij}\wedge w_{ki}+(w_{ki}-w_{kj})\wedge w_{ik}\right)
\quad.
\end{multlined}
\end{split}
\end{equation}
It follows that the quadratic form \eqref{eq:singlocus-def} together with the linear equation $t_i+t_j+t_K=1$ defines the singular locus of $p$.
Note that, if $\dim V=n$, then $V\wedge V$ is of dimension $n(n-1)/2$; so we have $n(n-1)/2$ defining polynomials.
In particular, there is generically no critical point of $p$ except in the case $n=2$ where they form a $1$-dimensional submanifold of $\Delta^2$.


\section{Rendering quadratic curves by B\'ezier curves}

\subsection{Definition}

Recall that a \emph{quadratic form} on $\mathbb R^n$ is nothing but a real homogeneous polynomial of degree $2$ with $n$ variables.
More precisely, it is a polynomial of the form
\begin{equation}
\label{eq:quad-form-alph}
q(x_1,\dots,x_n)
= \sum_{i=1}^n A_ix_i^2+2\sum_{1\le i<j\le n} B_{ij}x_ix_j
\quad.
\end{equation}
For a quadratic form $q$ on $\mathbb R^n$, we define the associated matrix $M_q$ to be
\[
\begin{bmatrix}
A_1 & B_{12} & \cdots & B_{1n} \\
B_{12} & A_2 & \cdots & B_{2n} \\
\vdots & \vdots & \ddots & \vdots \\
B_{1n} & B_{2n} & \cdots & A_n
\end{bmatrix}
\quad.
\]
Then, we can write
\begin{equation}
\label{eq:quad-form-mat}
q(x_1,\dots,x_n)
= \vec x^{\mathsf T} M_q\vec x
=
\begin{bmatrix}
x_1 & \cdots & x_n
\end{bmatrix}
M_q
\begin{bmatrix}
x_1 \\ \vdots \\ x_n
\end{bmatrix}
\quad.
\end{equation}

\begin{definition}
A quadratic form $q$ on $\mathbb R^n$ is said to be \emph{non-degenerate} (resp. \emph{degenerate}) if $\det M_q\neq 0$ (resp. $\det M_q=0$).
\end{definition}

\begin{lemma}
If $q$ is a non-degenerate quadratic form on $\mathbb R^n$, then for the map
\[
f_q:\mathbb R^{n-1}\to\mathbb R\ ;\quad (y_1,\dots,y_{n-1}) \mapsto q(y,\dots ,y_{n-1},1)
\quad,
\]
$0\in\mathbb R$ is a regular value of $f_q$.
\end{lemma}
\begin{proof}
We prove the contraposition; in particular, we show that if $\vec y=(y_1,\dots,y_{n-1})\in\mathbb R^{n-1}$ is a critical point of $f_q$ with $f_q(\vec y)=0$, then the vector $(y_1,\dots,y_{n-1},1)$ belongs to the kernel of $M_q$.

The direct computation shows that the total derivative $df:\mathbb R^{n-1}\to T^\ast\mathbb R^{n-1}$ is given by
\[
df_q(y_1,\dots,y_{n-1})
= \sum_{i=1}^{n-1}\left(\vec e^{(i)\mathsf T} M_q\widehat y+\widehat y^{\mathsf T} M_q e_i\right) dy_i \\
= 2 \sum_{i-1}^{n-1} \vec e^{(i)\mathsf T}M_q\widehat y\, dy_i
\quad,
\]
here $\widehat y=(y_1,\dots,y_{n-1},1)$ and $\vec e^{(i)}=(0,\dots,\overset{\substack{i\\\smash\smile}}{1},\dots,0)$ seen as column vectors.
Under the canonical identification $T^\ast\mathbb R^{n-1}\cong \mathbb R^{n-1}\times\mathbb R^{n-1}$, it turns out that $df$ can be seen as the map given by
\begin{equation}
\label{eq:prf:fq-diff}
df_q(y_1,\dots,y_{n-1})
= 2M_q'\vec y+ 2v_q
\quad,
\end{equation}
where $\vec y=(y_1,\dots,y_{n-1},0)$ and $M_q'$ and $v_q$ are defined so that
\[
M_q =
\left[
\begin{array}{c|c}
  M_q' & v_q  \rule[-1.5ex]{0pt}{4ex}\\\hline\rule[-1.5ex]{0pt}{4ex}
  v_q^{\mathsf T} & A_n
\end{array}
\right]
\quad.
\]
On the other hand, we have
\[
\begin{split}
f_q(y_1,\dots,y_{n-1})
&= q(y_1,\dots,y_{n-1},1) \\
&= \widehat y^{\mathsf T} M_q\widehat y \\
&= \vec y^{\mathsf T} M_q'\vec y + 2v_q^{\mathsf T}\vec y + A_n
\quad.
\end{split}
\]
Hence, for $(y_1,\dots,y_{n-1})\in\mathbb R^{n-1}$ with $df(y_1,\dots,y_{n-1})=0$, we have
\begin{equation}
\label{eq:prf:fq-critv}
f_q(y_1,\dots,y_{n-1})
= v_q^{\mathsf T}\vec y + A_n
\quad.
\end{equation}
Combining \eqref{eq:prf:fq-diff} and \eqref{eq:prf:fq-critv}, we obtain $M_q\widehat y=0$ provided $df_q(y_1,\dots,y_{n-1})=f_q(y_1,\dots,y_{n-1})=0$.
This is exactly what we want to show.
\end{proof}

\begin{definition}
A \emph{(real) quadratic curve} is a curve $C$ in $\mathbb R^2$ such that
\[
C=\{(x,y)\in\mathbb R^2\mid q(x,y,1)=0\}
\]
for a quadratic form $q(x,y,z)$ on $\mathbb R^3$.
\end{definition}

\begin{remark}
It is easily seen that a quadratic form actually defines an algebraic function on the projective space $\mathbb P^n$.
One may sometimes use the word ``quadratic curves'' to refer algebraic subsets of the projective space $\mathbb P^2$ defined by quadratic forms of three variables.
In such cases, ``quadratic curves'' are always compact while ours might not.
\end{remark}

\subsection{Intersections with segments}

We next discuss the intersections of quadratic curves with line segments.
Recall that, for two points $v,w\in\mathbb R^{n-1}$ in the Euclidean space, the line segment connecting them is given as the function $\gamma_{vw}:[0,1]\to\mathbb R^{n-1}$ with
\[
\gamma_{vw}(t)\coloneqq (1-t)v + tw
\quad.
\]

\begin{lemma}
Let $q$ be a quadratic form on $\mathbb R^n$ and $v,w\in\mathbb R^{n-1}$.
Then, the line segment $\gamma_{vw}$ intersects with the set
\[
Q\coloneqq \left\{(y_1,\dots,y_{n-1})\mid q(y_1,\dots,y_{n-1},1)=0\right\}
\]
at a parameter $t\in[0,1]$ if and only if $t$ is the solution of the following equation:
\[
\left((w-v)^{\mathsf T}M'_q(w-v)\right)t^2
+ 2\left((w-v)^{\mathsf T}(M'_q v + v_q)\right)t
+ \widehat v^{\mathsf T}M_q \widehat v
= 0
\quad,
\]
where $M_q$, $M'_q$, and $v_q$ are defined as in the previous section.
\end{lemma}
\begin{proof}
We put $f_q(y_1,\dots,y_{n-1})\coloneqq q(y_1,\dots,y_{n-1},1)$, then we have to solve the equation $f_q(\gamma_{vw}(t))=0$.
Since $f_q$ is a polynomial of degree at most $2$, computing the Taylor series of the composition $f_q\gamma_{vw}$, we obtain
\[
\begin{split}
&[f_q\circ\gamma_{vw}](t) \\
&= f_q(v+t(w-v)) \\
&= f_q(v)
+ t\cdot\left(\left.\frac{d}{dt}\right|_{t=0} f_q(v+t(w-v))\right)
+ \frac{t^2}{2}\left(\left.\frac{d^2}{dt^2}\right|_{t=0} f_q(v+t(w-v))\right)
\quad.
\end{split}
\]
On the other hand, we have
\[
(\gamma_{vw})_\ast\left(\frac{d}{dt}\right)
= \sum_{i=1}^{n-1} (v_i-w_i)\frac\partial{\partial y_i}
\quad.
\]
Thus, combining with the equation \eqref{eq:prf:fq-diff}, one obtains the result.
\end{proof}

\subsection{Algorithm}

Fix a quadratic form $q(x,y,z)$ on $\mathbb R^3$.
In this section, we discuss an algorithm rendering the quadratic curve $C=\{(x,y)\mid q(x,y,1)=0\}$ with quadratic B\'ezier curvees.
The following is a pseudo-code for our approach:
\begin{algorithmic}[1]
\Procedure{Render-Quad-Curve}{$q$:\:quadratic form, $X$:\:canvas}
\State $T\gets$ sufficiently fine triangualation of $X$
\ForAll{$\Delta\in T$}
\State $S\gets$ \Call{Segments}{$q$,$\Delta$}
\ForAll{$(v,w)\in S$}
\State $c\gets \operatorname{solve}(d_vq(x-v)=d_wq(x-w)=0)$
\State $\operatorname{qbezier}(v,c,w)$
\EndFor
\EndFor
\EndProcedure
\end{algorithmic}
In the function \textsc{Segments}, we compute the pairs $(v,w)$ such that
\begin{itemize}
  \item $v,w\in\partial\Delta\cap\{f_q=0\}$;
  \item $v$ and $w$ belong to the same connected component of $\Delta\cap\{f_q=0\}$.
\end{itemize}
We implement it as follows.
Note that, for two totally ordered sets $P$ and $Q$, we denote by $P\star Q$ the join of them.
\begin{algorithmic}[1]
\Procedure{Segments}{$q$:\:quadratic form, $\Delta$:\:$2$-dim.~polygon}
\State $I\gets\varnothing$ as a totally ordered set
\ForAll{$v\in\operatorname{Vertex}(\Delta)$}
\If{$q(v)=0$}
\State $I\gets I\star\{v\}$
\EndIf
\EndFor
\ForAll{$e\in\operatorname{Edge}(\Delta)$}
\State $I\gets I\star(\operatorname{Interior}(e)\cap \{f_q=0\})$
\EndFor
\State \Return \Call{Pairs}{$q$, $I$}
\EndProcedure

\Procedure{Pairs}{$q$:\:a quadratic form, $I$:\:a set of points in $\{f_q=0\}$}
\If{$\#I\le 1$}
\Comment{Nothing can form a pair}
\State \Return $\varnothing$
\ElsIf{$\#I=2$}
\Comment{A connected pair is found}
\State Let $\{(v,w)\}\gets I$
\State \Return $\{(v,w)\}$
\EndIf
\State $v\gets\operatorname{Head}(I)$
\Comment{Fix a reference point}
\State $I_+\gets \{w\in I\mid d_vq(w-v)>0\}$
\Comment{the subset of points lying on the ``right'' of $v$}
\State $I_-\gets \{w\in I\mid d_vq(w-v)<0\}$
\Comment{the subset of points lying on the ``left'' of $v$}
\State \Return \Call{Pairs}{$q$, $I_+\star\{v\}$} $\cup$ \Call{Pairs}{$q$, $I_-\star\{v\}$}
\Comment{Make pairs on the left and the right respectively}
\EndProcedure
\end{algorithmic}

\end{document}
